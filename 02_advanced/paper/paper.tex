% VLDB template version of 2020-08-03 enhances the ACM template, version 1.7.0:
% https://www.acm.org/publications/proceedings-template
% The ACM Latex guide provides further information about the ACM template

\PassOptionsToPackage{dvipsnames}{xcolor}
\documentclass[sigconf, nonacm]{acmart}

\usepackage{lipsum}
\usepackage{tikz}
\usepackage{pgfplots}
\usepackage{pgfplotstable}
\pgfplotsset{compat=1.18}
\usepackage{contour}

%% The following content must be adapted for the final version
% paper-specific
\newcommand\vldbdoi{XX.XX/XXX.XX}
\newcommand\vldbpages{XXX-XXX}
% issue-specific
\newcommand\vldbvolume{14}
\newcommand\vldbissue{1}
\newcommand\vldbyear{2020}
% should be fine as it is
\newcommand\vldbauthors{\authors}
\newcommand\vldbtitle{\shorttitle}
% leave empty if no availability url should be set
\newcommand\vldbavailabilityurl{URL_TO_YOUR_ARTIFACTS}
% whether page numbers should be shown or not, use 'plain' for review versions, 'empty' for camera ready
\newcommand\vldbpagestyle{plain}

\begin{document}
\title{A Relational Model of Data for Large Shared Data Banks}

%%
%% The "author" command and its associated commands are used to define the authors and their affiliations.
\author{Ben Trovato}
\affiliation{%
  \institution{Institute for Clarity in Documentation}
  \streetaddress{P.O. Box 1212}
  \city{Dublin}
  \state{Ireland}
  \postcode{43017-6221}
}
\email{trovato@corporation.com}

\author{Lars Th{\o}rv{\"a}ld}
\orcid{0000-0002-1825-0097}
\affiliation{%
\institution{The Th{\o}rv{\"a}ld Group}
\streetaddress{1 Th{\o}rv{\"a}ld Circle}
\city{Hekla}
\country{Iceland}
}
\email{larst@affiliation.org}

\author{Valerie B\'eranger}
\orcid{0000-0001-5109-3700}
\affiliation{%
  \institution{Inria Paris-Rocquencourt}
  \city{Rocquencourt}
  \country{France}
}
\email{vb@rocquencourt.com}

\author{J\"org von \"Arbach}
\affiliation{%
  \institution{University of T\"ubingen}
  \city{T\"ubingen}
  \country{Germany}
}
\email{jaerbach@uni-tuebingen.edu}
\email{myprivate@email.com}
\email{second@affiliation.mail}

\author{Wang Xiu Ying}
\author{Zhe Zuo}
\affiliation{%
  \institution{East China Normal University}
  \city{Shanghai}
  \country{China}
}
\email{firstname.lastname@ecnu.edu.cn}

\author{Donald Fauntleroy Duck}
\affiliation{%
  \institution{Scientific Writing Academy}
  \city{Duckburg}
  \country{Calisota}
}
\affiliation{%
  \institution{Donald's Second Affiliation}
  \city{City}
  \country{country}
}
\email{donald@swa.edu}

%%
%% The abstract is a short summary of the work to be presented in the
%% article.
\begin{abstract}
  \lipsum[1]
\end{abstract}

\maketitle

%%% do not modify the following VLDB block %%
%%% VLDB block start %%%
\pagestyle{\vldbpagestyle}
\begingroup\small\noindent\raggedright\textbf{PVLDB Reference Format:}\\
\vldbauthors. \vldbtitle. PVLDB, \vldbvolume(\vldbissue): \vldbpages, \vldbyear.\\
\href{https://doi.org/\vldbdoi}{doi:\vldbdoi}
\endgroup
\begingroup
\renewcommand\thefootnote{}\footnote{\noindent
  This work is licensed under the Creative Commons BY-NC-ND 4.0 International License. Visit \url{https://creativecommons.org/licenses/by-nc-nd/4.0/} to view a copy of this license. For any use beyond those covered by this license, obtain permission by emailing \href{mailto:info@vldb.org}{info@vldb.org}. Copyright is held by the owner/author(s). Publication rights licensed to the VLDB Endowment. \\
  \raggedright Proceedings of the VLDB Endowment, Vol. \vldbvolume, No. \vldbissue\ %
  ISSN 2150-8097. \\
  \href{https://doi.org/\vldbdoi}{doi:\vldbdoi} \\
}\addtocounter{footnote}{-1}\endgroup
%%% VLDB block end %%%

%%% do not modify the following VLDB block %%
%%% VLDB block start %%%
\ifdefempty{\vldbavailabilityurl}{}{
  \vspace{.3cm}
  \begingroup\small\noindent\raggedright\textbf{PVLDB Artifact Availability:}\\
  The source code, data, and/or other artifacts have been made available at \url{\vldbavailabilityurl}.
  \endgroup
}
%%% VLDB block end %%%

\begin{figure*}[t]
  \centering
  \begin{tikzpicture}[%
      rounded corners=0.3pt,                      %% soften the edges a bit
    ]
    \begin{axis}[%
        width=\textwidth,                         %% make full use of the entire page width
        height=8cm,                               %% set the height to something reasonable
        ybar,                                     %% plot as bart chart
        bar width=0.2cm,                          %% set the width of the bars, pgfplots defaults are unsightly
        xlabel={TPC-H Query},                     %% we're plotting TPC-H queries on the x-axis
        ylabel={Runtime [ms]},                    %% be explicit about the y-axis (including units)
        nodes near coords={\contour{white}{\pgfmathprintnumber\pgfplotspointmeta}},  %% show the exact runtime on and include a contour to make them pop
        nodes near coords align={vertical},       %% place runtimes on the top of the bars
        ymode=log,                                %% use a logarithmic scale for y-axis
        ymin=0,                                   %% start the y-axis at 0ms
        ymajorgrids=true,                         %% show horizontal grid lines
        xtick={1,2,...,22},                       %% we're plotting all 22 queries
        xticklabel={Q\pgfmathprintnumber{\tick}}, %% use Q1, Q2, ..., Q22 as labels
        enlarge x limits=0.05,                    %% stretch the x-axis a bit, otherwise some bars are cut off
        xtick style={draw=none},                  %% don't draw the actual x-ticks, but leave their labels untouched
        minor xtick={0.5,1.5,...,22.5},           %% add some minor x-ticks and...
        xminorgrids=true,                         %% ...draw some gridlines on them
        legend style={%
            at={(0.5,-0.2)},                      %% place the legend below the plot
            anchor=north,                         %% anchor it to the north, otherwise it will collide with the x-ticks
            legend columns=-1                     %% use as many columns as necessary
          },
      ]

      \addplot[  %% plot the DuckDB runtimes from the CSV file
        draw=BurntOrange,
        fill=BurntOrange!40!white,
      ] table [
          col sep=comma,
          x expr=\coordindex + 1,
          y expr=\thisrow{runtime} * 1000 %% convert to ms since DuckDB measures in seconds!
        ] {../data/duckdb.csv};

      \addplot[  %% plot the PostgreSQL runtimes from the CSV file
        draw=RoyalBlue,
        fill=RoyalBlue!40!white,
      ] table [
          col sep=comma,
          x expr=\coordindex + 1,
          y=runtime
        ] {../data/postgres.csv};

      \legend{DuckDB, PostgreSQL}
    \end{axis}
  \end{tikzpicture}
  \caption{Comparing runtimes of DuckDB and PostgreSQL for all TPC-H queries at scale factor 1.}
  \label{fig:runtime-comparison}
\end{figure*}

%\clearpage

\bibliographystyle{ACM-Reference-Format}
\bibliography{sample}

\end{document}
\endinput
